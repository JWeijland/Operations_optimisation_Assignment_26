\documentclass[11pt,a4paper]{article}
\usepackage[utf8]{inputenc}
\usepackage[dutch]{babel}
\usepackage{amsmath}
\usepackage{amssymb}
\usepackage{geometry}
\usepackage{enumitem}
\usepackage{hyperref}
\usepackage{booktabs}
\usepackage{algorithm}
\usepackage{algpseudocode}

\geometry{margin=2.5cm}

\title{\textbf{Technische Uitleg\\Aircraft Landing Scheduling Model}}
\author{Operations Optimisation}
\date{\today}

\begin{document}

\maketitle

\section{Probleemomschrijving}

Het Aircraft Landing Scheduling probleem optimaliseert de planning van vliegtuiglandingen op meerdere landingsbanen. Het doel is om de totale kosten te minimaliseren die ontstaan wanneer vliegtuigen vroeg of laat landen ten opzichte van hun gewenste landingstijd, rekening houdend met veiligheidsbeperkingen zoals minimale separatietijden tussen opeenvolgende landingen.

\section{Wiskundige Formulering (MIP Model)}

\subsection{Indices en Verzamelingen}

\begin{itemize}[noitemsep]
    \item $i, j \in \{1, \ldots, P\}$: vliegtuigen
    \item $r \in \{1, \ldots, R\}$: landingsbanen
\end{itemize}

\subsection{Parameters}

Voor elk vliegtuig $i$:
\begin{align*}
E_i &: \text{Vroegste landingstijd (earliest landing time)} \\
T_i &: \text{Gewenste landingstijd (target landing time)} \\
L_i &: \text{Laatst mogelijke landingstijd (latest landing time)} \\
g_i &: \text{Kostprijs per tijdseenheid voor vroege landing (EUR/minuut)} \\
h_i &: \text{Kostprijs per tijdseenheid voor late landing (EUR/minuut)}
\end{align*}

Voor elk paar vliegtuigen $(i, j)$:
\begin{align*}
S_{ij} &: \text{Minimale separatietijd als vliegtuig } i \text{ voor } j \text{ landt}
\end{align*}

Overige parameters:
\begin{align*}
R &: \text{Aantal beschikbare landingsbanen} \\
M &: \text{Big-M constante voor disjunctieve constraints}
\end{align*}

\subsection{Beslissingsvariabelen}

\textbf{Continue variabelen:}
\begin{align*}
x_i &: \text{Werkelijke landingstijd van vliegtuig } i \\
\alpha_i &: \text{Tijd vóór target time (early deviation)} \\
\beta_i &: \text{Tijd ná target time (late deviation)}
\end{align*}

\textbf{Binaire variabelen:}
\begin{align*}
\delta_{ij} &\in \{0,1\}: \text{1 als vliegtuig } i \text{ vóór } j \text{ landt, 0 anders} \\
y_{ir} &\in \{0,1\}: \text{1 als vliegtuig } i \text{ op baan } r \text{ landt, 0 anders} \\
z_{ij} &\in \{0,1\}: \text{1 als vliegtuigen } i \text{ en } j \text{ op dezelfde baan landen}
\end{align*}

\subsection{Doelfunctie}

\begin{equation}
\min \sum_{i=1}^{P} \left( g_i \cdot \alpha_i + h_i \cdot \beta_i \right)
\end{equation}

De doelfunctie minimaliseert de totale kosten van vroege en late landingen over alle vliegtuigen.

\subsection{Constraints}

\textbf{1. Tijdvenster constraints:}
\begin{equation}
E_i \leq x_i \leq L_i \qquad \forall i \in \{1, \ldots, P\}
\end{equation}

\textbf{2. Target deviation constraints:}
\begin{equation}
x_i = T_i - \alpha_i + \beta_i \qquad \forall i \in \{1, \ldots, P\}
\end{equation}

\textbf{3. Separatie constraints (enkele landingsbaan, $R=1$):}

Voor alle $i < j$:
\begin{align}
x_j &\geq x_i + S_{ij} - M(1 - \delta_{ij}) \\
x_i &\geq x_j + S_{ji} - M \cdot \delta_{ij}
\end{align}

De Big-M techniek zorgt ervoor dat exact één van deze constraints actief is, afhankelijk van de volgorde.

\textbf{4. Separatie constraints (meerdere landingsbanen, $R > 1$):}

Voor alle $i < j$:
\begin{align}
x_j &\geq x_i + S_{ij} - M(2 - z_{ij} - \delta_{ij}) \\
x_i &\geq x_j + S_{ji} - M(1 + \delta_{ij} - z_{ij})
\end{align}

Separatie geldt alleen als $z_{ij} = 1$ (beide vliegtuigen op dezelfde baan).

\textbf{5. Landingsbaan toewijzing (alleen bij $R > 1$):}
\begin{equation}
\sum_{r=1}^{R} y_{ir} = 1 \qquad \forall i \in \{1, \ldots, P\}
\end{equation}

\textbf{6. Same runway indicator (alleen bij $R > 1$):}
\begin{equation}
z_{ij} \geq y_{ir} + y_{jr} - 1 \qquad \forall i < j, \; \forall r \in \{1, \ldots, R\}
\end{equation}

Als beide vliegtuigen op baan $r$ zitten ($y_{ir} = y_{jr} = 1$), dan moet $z_{ij} = 1$.

\subsection{Big-M Berekening}

De Big-M waarde moet voldoende groot zijn om constraints effectief te deactiveren:
\begin{equation}
M = \max_{i} L_i + \max_{i,j} S_{ij} + 100
\end{equation}

\section{Greedy Heuristiek}

\subsection{Constructie Fase}

\begin{algorithm}[H]
\caption{Greedy Constructive Heuristic}
\begin{algorithmic}[1]
\State Sorteer vliegtuigen op target time: $\text{sort}(\{1, \ldots, P\}, \text{key} = T_i)$
\For{elk vliegtuig $i$ in gesorteerde volgorde}
    \State $\text{best\_cost} \gets \infty$
    \For{elke landingsbaan $r \in \{1, \ldots, R\}$}
        \State $t \gets \text{FindEarliestFeasibleTime}(i, r)$
        \State $\text{cost} \gets g_i \cdot \max(0, T_i - t) + h_i \cdot \max(0, t - T_i)$
        \If{$\text{cost} < \text{best\_cost}$}
            \State $\text{best\_cost} \gets \text{cost}$
            \State $\text{best\_runway} \gets r$
            \State $\text{best\_time} \gets t$
        \EndIf
    \EndFor
    \State Wijs vliegtuig $i$ toe aan baan $\text{best\_runway}$ op tijd $\text{best\_time}$
\EndFor
\end{algorithmic}
\end{algorithm}

\textbf{Tijdscomplexiteit:} $O(P^2 \cdot R)$ waarbij $P$ = aantal vliegtuigen, $R$ = aantal landingsbanen.

\subsection{FindEarliestFeasibleTime Functie}

Deze functie vindt de vroegst haalbare landingstijd voor vliegtuig $i$ op baan $r$:

\begin{algorithm}[H]
\caption{FindEarliestFeasibleTime($i$, $r$)}
\begin{algorithmic}[1]
\State $t_{\text{candidate}} \gets T_i$ \Comment{Start bij target time}
\State $S_r \gets$ lijst van reeds geplande vliegtuigen op baan $r$
\If{$S_r = \emptyset$}
    \State \Return $\max(E_i, T_i)$
\EndIf
\While{$t_{\text{candidate}} \leq L_i$}
    \State $\text{feasible} \gets \text{true}$
    \For{elk vliegtuig $j \in S_r$}
        \If{$x_j \leq t_{\text{candidate}}$} \Comment{$j$ landt voor $i$}
            \If{$t_{\text{candidate}} < x_j + S_{ji}$}
                \State $t_{\text{candidate}} \gets x_j + S_{ji}$
                \State $\text{feasible} \gets \text{false}$
                \State \textbf{break}
            \EndIf
        \Else \Comment{$i$ zou voor $j$ landen}
            \If{$x_j < t_{\text{candidate}} + S_{ij}$}
                \State $t_{\text{candidate}} \gets x_j + 0.01$
                \State $\text{feasible} \gets \text{false}$
                \State \textbf{break}
            \EndIf
        \EndIf
    \EndFor
    \If{$\text{feasible}$ \textbf{and} $t_{\text{candidate}} \geq E_i$}
        \State \Return $t_{\text{candidate}}$
    \EndIf
\EndWhile
\State \Return $L_i$ \Comment{Fallback}
\end{algorithmic}
\end{algorithm}

\subsection{Verbetering Fase}

Na de constructie wordt de oplossing iteratief verbeterd:
\begin{enumerate}[noitemsep]
    \item Voor elk vliegtuig: probeer landingstijd dichter bij $T_i$ te brengen
    \item Accepteer alleen aanpassingen die kosten verlagen
    \item Valideer haalbaarheid met separatie-eisen
    \item Maximum 10 iteraties
\end{enumerate}

\subsection{Multi-Start Strategie}

De multi-start heuristiek probeert verschillende sorteervolgorden:
\begin{itemize}[noitemsep]
    \item Sorteren op $T_i$ (target time)
    \item Sorteren op $E_i$ (appearance time)
    \item Sorteren op $L_i - E_i$ (tijdvenster grootte)
    \item Sorteren op $g_i + h_i$ (totale penalty)
    \item Sorteren op $-g_i$ (early penalty, aflopend)
\end{itemize}

De beste oplossing over alle starts wordt geretourneerd.

\section{Implementatie Details}

\subsection{Wake Turbulence Categorieën}

Het model gebruikt realistische ICAO wake turbulence categorieën:

\begin{table}[h]
\centering
\begin{tabular}{@{}lcccc@{}}
\toprule
\textbf{Categorie} & \textbf{Voorbeelden} & \textbf{$g_i$} & \textbf{$h_i$} & \textbf{Separatie} \\
\midrule
Heavy (H) & B747, B777, A330 & 100 EUR/min & 200 EUR/min & 90--180s \\
Medium (M) & B737, A320, E190 & 60 EUR/min & 150 EUR/min & 60--120s \\
Light (L) & Citation, Phenom & 30 EUR/min & 80 EUR/min & 60s \\
\bottomrule
\end{tabular}
\caption{Vliegtuigcategorieën en parameters}
\end{table}

\subsection{Separatie Matrix}

De separatietijd $S_{ij}$ hangt af van de wake categorieën van vliegtuigen $i$ en $j$:

\begin{table}[h]
\centering
\begin{tabular}{@{}lccc@{}}
\toprule
\textbf{$i$ (leading)} & \textbf{$j$ = H} & \textbf{$j$ = M} & \textbf{$j$ = L} \\
\midrule
H & 90s & 120s & 180s \\
M & 60s & 60s & 120s \\
L & 60s & 60s & 60s \\
\bottomrule
\end{tabular}
\caption{Separatietijden $S_{ij}$ (in seconden)}
\end{table}

Opmerking: Zwaardere vliegtuigen veroorzaken meer wake turbulence, waardoor lichtere vliegtuigen langer moeten wachten.

\subsection{Realistische Scenario's}

\textbf{Schiphol Evening Rush (18:00--20:00):}
\begin{itemize}[noitemsep]
    \item Piekuur concentratie: 50--90\% van vliegtuigen arriveert tussen 19:00--19:30
    \item Tijdseenheid: minuten vanaf 18:00
    \item Vliegtuigmix: 30\% heavy, 60\% medium, 10\% light
    \item Tijdvensters variëren per type (heavy heeft meer brandstof reserve)
\end{itemize}

\section{Oplossingsmethoden}

\subsection{Methode 1: Greedy Heuristiek (Snel, Approximate)}

\subsubsection{Hoe werkt de heuristiek?}

De greedy heuristiek construeert een oplossing stap voor stap door vliegtuigen één voor één in te plannen volgens een vooraf bepaalde volgorde.

\textbf{Hoofdidee:} Plan elk vliegtuig op de best mogelijke tijd en landingsbaan, gegeven de reeds geplande vliegtuigen.

\subsubsection{Stapsgewijze Uitvoering}

\textbf{Stap 1: Sorteer vliegtuigen}
\begin{equation}
\text{Volgorde} = \text{sort}(\{1, \ldots, P\}, \text{key} = T_i)
\end{equation}

Vliegtuigen worden gesorteerd op target landing time. Dit is een greedy keuze: vliegtuigen die eerder willen landen krijgen voorrang.

\textbf{Stap 2: Voor elk vliegtuig $i$ (in volgorde):}

Voor elke landingsbaan $r \in \{1, \ldots, R\}$:

\begin{enumerate}[noitemsep]
    \item \textbf{Vind vroegst haalbare tijd} $t^*_r$ op baan $r$:
    \begin{equation}
    t^*_r = \max\left(E_i, \; T_i, \; \max_{j \in S_r} (x_j + S_{ji})\right)
    \end{equation}
    waarbij $S_r$ de set van reeds geplande vliegtuigen op baan $r$ is.

    \item \textbf{Bereken kosten} voor landing op tijd $t^*_r$:
    \begin{equation}
    C_r = \begin{cases}
    g_i \cdot (T_i - t^*_r) & \text{als } t^*_r < T_i \quad \text{(vroeg)} \\
    0 & \text{als } t^*_r = T_i \quad \text{(on-time)} \\
    h_i \cdot (t^*_r - T_i) & \text{als } t^*_r > T_i \quad \text{(laat)}
    \end{cases}
    \end{equation}

    \item Als $t^*_r > L_i$, markeer baan $r$ als niet-haalbaar.
\end{enumerate}

\textbf{Stap 3: Kies beste baan}
\begin{equation}
r^* = \argmin_{r \in \{1,\ldots,R\}} C_r
\end{equation}

Wijs vliegtuig $i$ toe aan baan $r^*$ met landingstijd $x_i = t^*_{r^*}$.

\subsubsection{Vroegst Haalbare Tijd Algoritme}

Voor een gegeven vliegtuig $i$ en baan $r$, vind de vroegst haalbare landingstijd:

\begin{algorithm}[H]
\caption{FindEarliestFeasibleTime($i$, $r$)}
\begin{algorithmic}[1]
\State $t \gets T_i$ \Comment{Start bij target time}
\State $S_r \gets$ \{vliegtuigen reeds gepland op baan $r$\}
\If{$S_r = \emptyset$}
    \State \Return $\max(E_i, T_i)$ \Comment{Lege baan: land op target}
\EndIf
\While{$t \leq L_i$} \Comment{Blijf zoeken binnen tijdvenster}
    \State $\text{conflict} \gets \text{false}$
    \For{$j \in S_r$} \Comment{Check alle reeds geplande vliegtuigen}
        \If{$x_j \leq t$} \Comment{Vliegtuig $j$ landt vóór $i$}
            \If{$t < x_j + S_{ji}$} \Comment{Te weinig separatie}
                \State $t \gets x_j + S_{ji}$ \Comment{Verschuif naar na separatie}
                \State $\text{conflict} \gets \text{true}$
                \State \textbf{break}
            \EndIf
        \Else \Comment{Vliegtuig $i$ zou vóór $j$ landen}
            \If{$x_j < t + S_{ij}$} \Comment{Te weinig separatie}
                \State $t \gets x_j + 0.01$ \Comment{Probeer net na $j$}
                \State $\text{conflict} \gets \text{true}$
                \State \textbf{break}
            \EndIf
        \EndIf
    \EndFor
    \If{$\neg\text{conflict}$ \textbf{and} $t \geq E_i$}
        \State \Return $t$ \Comment{Haalbare tijd gevonden!}
    \EndIf
\EndWhile
\State \Return $L_i$ \Comment{Geen haalbare tijd: fallback}
\end{algorithmic}
\end{algorithm}

\subsubsection{Verbeteringsfase (Local Search)}

Na de constructiefase wordt de oplossing iteratief verbeterd:

\begin{algorithm}[H]
\caption{Improve Solution}
\begin{algorithmic}[1]
\For{$\text{iter} = 1$ \textbf{to} $10$}
    \State $\text{improved} \gets \text{false}$
    \For{elk vliegtuig $i$}
        \State $t_{\text{old}} \gets x_i$
        \State $C_{\text{old}} \gets$ kosten van $t_{\text{old}}$
        \State $t_{\text{new}} \gets$ probeer shift richting $T_i$ (±1 minuut)
        \If{$t_{\text{new}}$ is haalbaar \textbf{and} $C_{\text{new}} < C_{\text{old}}$}
            \State $x_i \gets t_{\text{new}}$
            \State $\text{improved} \gets \text{true}$
        \EndIf
    \EndFor
    \If{$\neg\text{improved}$}
        \State \textbf{break} \Comment{Lokaal optimum bereikt}
    \EndIf
\EndFor
\end{algorithmic}
\end{algorithm}

\subsubsection{Complexiteit Analyse}

\begin{itemize}[noitemsep]
    \item \textbf{Sorteren:} $O(P \log P)$
    \item \textbf{Constructie:} Voor elk vliegtuig ($P$), probeer elke baan ($R$), check alle geplande vliegtuigen ($P$): $O(P^2 \cdot R)$
    \item \textbf{Verbetering:} 10 iteraties, elk vliegtuig, check separaties: $O(10 \cdot P^2)$
    \item \textbf{Totaal:} $O(P^2 \cdot R)$
\end{itemize}

Voor $P = 40$ vliegtuigen en $R = 4$ banen: $\approx 6400$ operaties $\rightarrow$ zeer snel (milliseconden).

\subsection{Methode 2: MILP Solver (Traag, Optimaal)}

\subsubsection{Hoe werkt de MILP optimalisatie?}

De MILP solver zoekt de \textit{globaal optimale} oplossing door systematisch de oplossingsruimte te verkennen met behulp van branch-and-bound.

\textbf{Hoofdidee:} Formuleer het probleem als Mixed Integer Linear Program en gebruik een gespecialiseerde solver (CBC, Gurobi, CPLEX).

\subsubsection{Branch-and-Bound Algoritme}

\textbf{Stap 1: LP Relaxatie}

Los het probleem op waarbij binaire variabelen $\delta_{ij}, y_{ir}, z_{ij} \in [0,1]$ continu mogen zijn:

\begin{equation}
\text{LP-relaxatie: } \min \sum_{i=1}^{P} (g_i \alpha_i + h_i \beta_i) \quad \text{s.t. alle constraints met } \delta_{ij}, y_{ir}, z_{ij} \in [0,1]
\end{equation}

Dit geeft een \textit{ondergrens} (lower bound) op de optimale kosten: $LB_{\text{LP}}$.

\textbf{Stap 2: Branching}

Kies een fractionale binaire variabele, bijvoorbeeld $\delta_{12} = 0.7$ (niet geheel):

\begin{itemize}
    \item \textbf{Branch 1:} Los op met $\delta_{12} = 0$ (vliegtuig 2 landt vóór 1)
    \item \textbf{Branch 2:} Los op met $\delta_{12} = 1$ (vliegtuig 1 landt vóór 2)
\end{itemize}

Dit creëert een \textbf{branch tree}:

\begin{verbatim}
                   LP-relaxation (LB = 1250.3)
                   δ_12 = 0.7 (fractional)
                          /\
                         /  \
              δ_12 = 0  /    \  δ_12 = 1
                       /      \
            (LB = 1280.5)    (LB = 1300.2)
                 /\              /\
               ...             ...
\end{verbatim}

\textbf{Stap 3: Bounding}

Bij elke node:
\begin{enumerate}[noitemsep]
    \item Los LP relaxatie op $\rightarrow$ lower bound $LB$
    \item Als $LB \geq UB$ (huidige beste oplossing), \textbf{prune} deze branch
    \item Als alle binaire variabelen geheel zijn $\rightarrow$ haalbare oplossing gevonden
    \item Update upper bound $UB$ als nieuwe oplossing beter is
\end{enumerate}

\textbf{Stap 4: Terminatie}

Stop wanneer:
\begin{itemize}[noitemsep]
    \item Alle branches zijn onderzocht of gepruned
    \item Time limit bereikt (50 seconden in onze implementatie)
    \item Optimality gap $\leq$ 1\%: $\frac{UB - LB}{LB} \times 100\% \leq 1\%$
\end{itemize}

\subsubsection{Cutting Planes}

Moderne MILP solvers gebruiken \textit{cutting planes} om de LP relaxatie te versterken:

\begin{itemize}[noitemsep]
    \item \textbf{Gomory cuts:} Verwijder fractionale oplossingen
    \item \textbf{Clique cuts:} Uit binaire conflicten
    \item \textbf{Cover cuts:} Uit knapsack constraints
\end{itemize}

Deze cuts voegen extra constraints toe die:
\begin{enumerate}[noitemsep]
    \item Geen haalbare gehele oplossingen verwijderen
    \item Wel fractionale oplossingen uitsluiten
    \item De lower bound verhogen (snellere convergentie)
\end{enumerate}

\subsubsection{Presolve Technieken}

Voor het branch-and-bound algoritme start, voert de solver \textit{presolve} uit:

\begin{enumerate}[noitemsep]
    \item \textbf{Variable fixing:} Als $x_i \geq E_i$ altijd geldt, verwijder redundante constraints
    \item \textbf{Constraint tightening:} Versterk bounds waar mogelijk
    \item \textbf{Redundancy elimination:} Verwijder constraints die altijd gelden
    \item \textbf{Coefficient reduction:} Vereenvoudig grote coëfficiënten
\end{enumerate}

\textbf{Voorbeeld:} Als $E_i = 10$ en alle $S_{ji} \geq 2$, dan geldt automatisch $x_i \geq 10$ en hoeft dit niet expliciet gecontroleerd.

\subsubsection{Heuristieken in MILP Solver}

Zelfs binnen de MILP solver worden heuristieken gebruikt:

\begin{itemize}[noitemsep]
    \item \textbf{Rounding heuristic:} Rond fractionale variabelen af naar 0 of 1
    \item \textbf{Diving heuristic:} Maak greedy keuzes in branch tree
    \item \textbf{RINS (Relaxation Induced Neighborhood Search):} Fix variabelen die in veel goede oplossingen dezelfde waarde hebben
\end{itemize}

Deze geven snel een goede upper bound $UB$, waardoor meer branches gepruned kunnen worden.

\subsubsection{Complexiteit}

\begin{itemize}[noitemsep]
    \item \textbf{Worst case:} $O(2^B)$ waarbij $B$ = aantal binaire variabelen
    \item Voor ons probleem: $B = O(P^2)$ (vanwege $\delta_{ij}$ variabelen)
    \item Bij $P = 40$: $B \approx 780$ binaire variabelen $\rightarrow 2^{780}$ theoretische nodes!
    \item \textbf{In praktijk:} Branch-and-bound + cutting planes + heuristics vinden optimum in $10^3$--$10^6$ nodes
\end{itemize}

\subsection{Vergelijking Heuristiek vs MILP}

\begin{table}[h]
\centering
\begin{tabular}{@{}lcc@{}}
\toprule
\textbf{Aspect} & \textbf{Greedy Heuristiek} & \textbf{MILP Solver} \\
\midrule
Oplossing & Approximate & Optimaal* \\
Tijdcomplexiteit & $O(P^2 \cdot R)$ & $O(2^{P^2})$ worst-case \\
Solve tijd (40 aircraft) & 0.01--0.1s & 10--50s \\
Kwaliteit (gap) & 5--15\% & 0\% (optimaal) \\
Garanties & Haalbaar & Haalbaar + optimaal \\
Schaalbaarheid & Tot 100+ aircraft & Tot 40--50 aircraft \\
Gebruik & Real-time, quick preview & Offline planning \\
\bottomrule
\multicolumn{3}{l}{\small *binnen time limit en gap tolerantie}
\end{tabular}
\caption{Vergelijking oplossingsmethoden}
\end{table}

\subsection{Solver Configuratie}

Het model gebruikt PuLP met CBC solver:
\begin{itemize}[noitemsep]
    \item Time limit: 50 seconden (configureerbaar)
    \item Optimality gap: 1\% (configureerbaar)
    \item Branch-and-bound voor binaire variabelen
    \item LP relaxatie voor continue variabelen
    \item Cutting planes: automatisch gegenereerd
    \item Presolve: enabled
\end{itemize}

\subsection{Vergelijking Heuristiek vs Optimal}

Performance metrics:
\begin{align*}
\text{Gap} &= \frac{\text{Cost}_{\text{heuristic}} - \text{Cost}_{\text{optimal}}}{\text{Cost}_{\text{optimal}}} \times 100\% \\
\text{Speedup} &= \frac{\text{Time}_{\text{optimal}}}{\text{Time}_{\text{heuristic}}}
\end{align*}

Typische resultaten:
\begin{itemize}[noitemsep]
    \item Gap: 5--15\% (heuristiek is meestal 5--15\% duurder)
    \item Speedup: 100--1000$\times$ (heuristiek is veel sneller)
    \item Heuristiek is altijd haalbaar (guaranteed feasibility)
\end{itemize}

\section{Validatie}

De oplossing wordt gevalideerd op:
\begin{enumerate}[noitemsep]
    \item \textbf{Tijdvensters:} $E_i \leq x_i \leq L_i$ voor alle $i$
    \item \textbf{Separatie:} $x_j \geq x_i + S_{ij}$ als $i$ vóór $j$ landt op dezelfde baan
    \item \textbf{Landingsbaan toewijzing:} Elk vliegtuig op precies één baan
    \item \textbf{Kosten berekening:} Correcte berekening van $\alpha_i$ en $\beta_i$
\end{enumerate}

\section{Output en Visualisatie}

Het model genereert:
\begin{itemize}[noitemsep]
    \item \textbf{Gantt charts:} Visuele weergave van landingsschema per baan
    \item \textbf{Cost breakdown:} Verdeling van kosten over vliegtuigen
    \item \textbf{Comparison plots:} Heuristiek vs optimale oplossing
    \item \textbf{Detailed tables:} CSV/Excel met complete oplossingsdetails
    \item \textbf{Runway analysis:} Impact van aantal landingsbanen op kosten
\end{itemize}

\end{document}
